\documentclass{isprs}
\usepackage{subfigure}
\usepackage{setspace}
\usepackage{geometry} % added 27-02-2014 Markus Englich
\geometry{a4paper, top=25mm, left=20mm, right=20mm, bottom=25mm, headsep=10mm, footskip=12mm} % added 27-02-2014 Markus Englich
%\usepackage{enumitem}

%\usepackage{isprs}
%\usepackage[perpage,para,symbol*]{footmisc}

%\renewcommand*{\thefootnote}{\fnsymbol{footnote}}



\begin{document}

\title{SPATIAL COGNITION IN TANGIBLE COMPUTING}

\author{
 B. A. Harmon\textsuperscript{a}\thanks{Corresponding author}
 , A. Petrasova\textsuperscript{a}, V. Petras\textsuperscript{a}, H. Mitasova\textsuperscript{a}, R. K. Meentemeyer\textsuperscript{a}}

\address
{
	\textsuperscript{a }Center for Geospatial Analytics, North Carolina State University - (baharmon, akratoc, vpetras, hmitaso, rkmeente)@ncsu.edu\\
%	\textsuperscript{b }College of Design, North Carolina State University - baharmon@ncsu.edu\\
%	\textsuperscript{c }Department of Marine, Earth, and Atmospheric Sciences, North Carolina State University - (akratoc, vpetras, hmitaso)@ncsu.edu
}

\commission{}{} %This field is optional.
\workinggroup{} %This field is optional.
\icwg{}   %This field is optional.
\ths{16} %ThS16

\abstract
{
Complex spatial forms like topography can be challenging to understand, much less intentionally shape, given the heavy cognitive load of visualizing and manipulating 3D form. 
This cognitive work can be offloaded onto computers through 3D geospatial modeling, analysis, and simulation.
Interacting with computers, however, can also be challenging 
requiring training and highly abstract thinking that adds a new cognitive burden. 
Tangible computing -- an emerging paradigm of human-computer interaction in which data is physically manifested so that users can feel it and directly manipulate it -- aims to offload this added cognitive work onto the body. 
We have designed Tangible Landscape, a tangible interface powered by an open source geographic information system (GRASS GIS), 
so that users can naturally shape topography and interact with simulated processes with their hands 
in order to make observations, generate and test hypotheses, and make inferences about scientific phenomena
in a rapid, iterative process. 
Conceptually Tangible Landscape couples a malleable physical model with a digital model of a landscape through an continuous cycle of 3D scanning, geospatial modeling, and projection. 
We ran a terrain modeling experiment with 39 participants to test whether tangible interfaces like this can effectively enhance spatial performance by offloading cognitive processes onto computers and our bodies.  We used topographic and morphometric parameters, differencing, hydrological simulation, and spatial statistics to quantitatively assess spatial performance. 
We found that Tangible Landscape generally enhanced 3D spatial performance, but future work is need to understand the role of cognition, affect, motivation, and metacognition in tangible computing. 
}

% ISPRS SHORT PROGRAM DESCRIPTION
% We have designed Tangible Landscape, a tangible interface powered by an open source geographic information system (GRASS GIS), that physically manifests data so that users can naturally shape topography and interact with simulated processes with their hands in order to make observations, generate and test hypotheses, and make inferences about scientific phenomena in a rapid, iterative process. We ran a terrain modeling experiment with 39 participants and found that tangible interfaces like this can effectively enhance spatial performance by offloading cognitive processes onto computers and our bodies. 


\keywords{Embodied cognition, spatial thinking, spatial performance, tangible user interfaces, user experiment, 3D}

\maketitle

% DRAFTS

% and also helped some participants better understand the dynamics of water flow.

% Spatiotemporal process like the flow of water over a landscape are even more challenging to understand and intentionally direct as they are dependent upon context, require multi-scale spatial thinking, and the simulation of forces like gravity and momentum. 

%Complex spatial forms like topography can be challenging to understand, much less intentionally shape, given the heavy cognitive load of visualizing and manipulating 3D form. 
%Spatiotemporal process like the flow of water over a landscape are even more challenging to understand and intentionally direct as they as dependent upon their context and thus require spatial thinking across multiple scales simultaneously and the simulation of forces like gravity and momentum. 
%This cognitive work can be offloaded onto computers through 3D geospatial modeling, analysis, and simulation.
%Interacting with computers, however, can also be challenging 
%requiring training and highly abstract thinking that adds a new cognitive burden. 
%Tangible computing -- an emerging paradigm of human-computer interaction in which data is physically manifested so that users can feel it and directly manipulate it -- aims to offload this added cognitive work onto the body. 
%We have designed Tangible Landscape, a tangible interface powered by an open source geographic information system (GRASS GIS), 
%to enable users to intuitively model, understand, and interact with scientific phenomena in real-time. 
%Conceptually Tangible Landscape couples a malleable physical model with a digital model of a landscape through an continuous cycle of 3D scanning, geospatial modeling, and projection. 
%Users can shape topography and interact with simulated processes with their hands 
%in order to make observations, generate and test hypotheses, and make inferences about scientific phenomena
%in a rapid, iterative process. 
%We ran a experiment to test whether tangible interfaces like this can effectively enhance spatial performance by offloading cognitive processes onto computers and our bodies. 
%We used geospatial analytics to quantitatively study how 
%different technologies mediate spatial thinking in a series of terrain modeling exercises. 
%We found that Tangible Landscape enhanced 3D spatial performance and
%for some elucidated the dynamics of water flow. 
%Future work is need to understand the role of cognition, affect, motivation, and metacognition in tangible computing. 


%Finding and implications: enhanced 3D spatial performance, (reduced anxiety and mystery vs. digital 3d modeling), initial lack of understanding of water flow process, sense of touch -- elucidation / demystification of water flow process, question of added cognitive load and frustration, design and technological issues

%Complex spatial forms and spatiotemporal processes can be challenging to understand, requiring spatial thinking across multiple scales simultaneously and the simulation of forces like gravity and momentum. Physical processes like the flow and dispersion of water are challenging to understand because they unfold in time and space, are historically contingent, are controlled by their context, and are driven by forces like gravity and momentum. The flow and dispersion of water is controlled by the morphological shape, gradient, and topology of the landscape / topography.

%Tangible user interfaces are based on the premise
%that embodied cognition in computing
%can enhance cognitive processes.
%However, the ways in which embodied cognition in computing
%transform spatial thinking have not yet been rigorously studied.
%We have co-designed Tangible Landscape --
%a continuous shape display 
%powered by a geographic information system --
%and used it to explore how 
%technology mediates spatial cognition 
%in a rigorous experiment. 
%In this terrain modeling experiment 
%we use geospatial analytics 
%%such as geomorphometry and geostatistics
%% supplemented by semi-structured interviews and direct observation
%to analyze how %quantitatively and qualitatively 
%visual computing with a GUI
%and tangible computing with a shape display
%mediate multidimensional spatial performance. 
%Our initial findings suggest that: 
%\begin{enumerate*}[label=\bfseries \arabic*.]
%\item digital sculpting via a GUI 
%is unintuitive, 
%%and visual ambiguity and misreadings 
%%lead to spatial misinterpretations,
%\item shape displays like Tangible Landscape can be intuitive, 
%enhance spatial performance, and
%%in novel ways
%enable rapid iteration and ideation,
%%enable rapid iterations of generative form-finding and critical analysis, 
%%and encourage creativity, 
%and
%\item different analytics encourage 
%significantly different modes of spatial thinking 
%and strategies for modeling. 
%\end{enumerate*}
%In conclusion 
%intuitive, kinaesthetic interaction 
%with useful computational analytics 
%can enhance spatial thinking.



\end{document}


